%!TEX root = report.tex

\section{Введение}

В данной работе предлагается рассмотреть систему репрессилатора  - три системы транскрипционных репрессора, которые не являются частью
каких-либо естественных биологических часов, чтобы построить колебательный сети. В роли программного обеспечения для анализа системы будет
использоваться пакет COPASI.

COPASI - это программа с открытым исходным кодом для создания и решения математических моделей биологических процессов,
таких как метаболические сети, сигнальные пути клеток, регуляторные сети, инфекционные заболевания и многое другое.

Цель работы - знакомство с пакетом COPASI на примере системы репрессилатора.

\section{Система}
Система состоит из 12 биохимических реакций и 7 реагентов.
Реагенты:
\begin{itemize}
  \item X, PX - мРНК и протеин гена LacI
  \item Y, PY - мРНК и протеин гена tetR
  \item Z, PZ - мРНК и протеин гена Cl
\end{itemize}

Реакции системы представлены на рисунке \ref{ris:reactions}.
\begin{figure}[!h]
\center{\includegraphics[width=1\linewidth,height=8cm]{reactions}}
\caption{Биохимические реакции системы репрессилатора}
\label{ris:reactions}
\end{figure}

C системой дифференциальных уравнений для нашей системы можно познакомиться на рисунке \ref{ris:equations}.
\begin{figure}[!h]
\center{\includegraphics[width=1\linewidth]{equations}}
\caption{Дифференциальные уравнения системы репрессилатора}
\label{ris:equations}
\end{figure}

\section{Анализ динамики системы}
Первоначально исследуем поведение системы на промежутке 1 секунда с размером интервала 0.01 секунда (Рис. \ref{ris:plot1}). В данном случаем
нам сложно что-то утверждать о поведении концентрации tetR и Cl.

\begin{figure}[!h]
\center{\includegraphics[width=1\linewidth,height=9cm]{plot1}}
\caption{Динамика поведения системы на промежутке 1 секунда}
\label{ris:plot1}
\end{figure}

Увеличим временной промежуток до 10 секунд (Рис. \ref{ris:plot2}). Теперь мы можем сравнивать между собой кривые. На графике отчётливо видно,
что мРНК и протеин для одного гена изменяются одинаково.

\begin{figure}[!h]
\center{\includegraphics[width=1\linewidth]{plot2}}
\caption{Динамика поведения системы на промежутке 10 секунд}
\label{ris:plot2}
\end{figure}

С увеличением временного промежутка до 50 секунд можно утверждать закономерности изменения концентраций мРНК и протеина(Рис. \ref{ris:plot3}).
То есть сначала вырабатывается мРНК и протеин tetR из Cl, далее из них получается мРНК и протеин LacI и в конце имеем мРНК и протеин Cl
после чего процесс повторяется.

\begin{figure}[!h]
\center{\includegraphics[width=1\linewidth]{plot3}}
\caption{Динамика поведения системы на промежутке 50 секунд}
\label{ris:plot3}
\end{figure}

\newpage
\section{Анализ состояния равновесия системы}
Пакет COPASI предлагает три подхода к поиску состояния равновесия:
\begin{enumerate}
  \item Метод Ньютона
  \begin{itemize}
    \item Быстрый
    \item Не гарантируется сходимость
  \end{itemize}
  \item Метод интегрирования
  \begin{itemize}
    \item Медленнее
    \item Гарантируется сходимость
  \end{itemize}
  \item Метод обратного интегрирования
  \begin{itemize}
    \item Используется в крайне редких случаях
  \end{itemize}
\end{enumerate}

С помощью метода Ньютона обнаружить состояние равновесие не удалось (Рис. \ref{ris:newton}).

\begin{figure}[!h]
\center{\includegraphics[width=1\linewidth]{newton}}
\caption{Результат поиска состояния равновесия с помощью метода Ньютона}
\label{ris:newton}
\end{figure}

Однако с помощью метода интегрирования удалось найти состояние равновесия (Рис. \ref{ris:int}).

\begin{figure}[!h]
\center{\includegraphics[width=1\linewidth]{int}}
\caption{Результат поиска состояния равновесия с интегрирования}
\label{ris:int}
\end{figure}

Теперь зная состояние равновесия, обновим модель и посмотрим динамику системы.
На рисунке \ref{ris:steady} можем наблюдать, что система находится в состоянии равновесия.

\begin{figure}[!h]
\center{\includegraphics[width=1\linewidth]{steady}}
\caption{Динамика системы в состоянии равновесия}
\label{ris:steady}
\end{figure}

\section{Анализ параметров системы}
Наша система имеет параметры - beta, alpha0, alpha1, K, k1, n. Посмотрим, как меняется поток реакции 12 в состоянии равновесия
при изменении параметра n. Как видно по рисунку \ref{ris:scan} никак не изменяется поток в зависимости от n.
К сожалению, попытки получить более красивые графики не увенчались успехом (проверялись другие параметры в остальных реакциях).

\begin{figure}[!h]
\center{\includegraphics[width=1\linewidth]{scan}}
\caption{Изменение потока в зависимости от параметра n}
\label{ris:scan}
\end{figure}

\section{Выводы}
Пакет COPASI предоставляет нам следующие возможности:
\begin{itemize}
  \item Работа с системами биохимических реакций. Возможность создавать новые, записывать в виде дифференциальных уравнений реакции и т.д.
  \item Анализ динамики системы на определённом временном промежутке с заданным интервалом
  \item Поиск состояния равновесия
  \item Анализ параметров системы
\end{itemize}
